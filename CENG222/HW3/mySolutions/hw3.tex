\documentclass[12pt]{article}
\usepackage[utf8]{inputenc}
\usepackage{float}
\usepackage{amsmath}
\usepackage{matlab-prettifier}
\usepackage{graphicx}
\usepackage{tikz}
\usepackage{pgfplots}


\usepackage[hmargin=3cm,vmargin=6.0cm]{geometry}
%\topmargin=0cm
\topmargin=-2cm
\addtolength{\textheight}{6.5cm}
\addtolength{\textwidth}{2.0cm}
%\setlength{\leftmargin}{-5cm}
\setlength{\oddsidemargin}{0.0cm}
\setlength{\evensidemargin}{0.0cm}

%misc libraries goes here

\begin{document}

\section*{Student Information } 
%Write your full name and id number between the colon and newline
%Put one empty space character after colon and before newline
Full Name :  Aytaç SEKMEN\\
Id Number :  2575983\\

% Write your answers below the section tags
\section*{Answer 1}

\subsection*{a)}
Probabilty I calculated through matlab: 0.0353. Which is relatively a low probabilty.

\subsection*{b)}
I calculated the expected as given in the book, by taking the mean of values in the array. And I found that the expected value is 238943.1774 which is acutally less than 300000. So it actually correlates with the probabilty I found in part-a.


\subsection*{c)} 
I calculated the standard deviation as given in the book, by taking the std of values in the array. And I found that the standard deviation is 32096.8436. My expected value is actually 61056.8226 tons behind the 300000 tons. Also 61056.8226> 32096.8436. Which then explains why the calculated probability I found in part-a is so small. This means, my estimator has high accuracy.


\subsection*{Related Documents} 
Codes:\\
\begin{lstlisting}[style=Matlab-editor]
%initial_data
alpha_01=2.326;
number_montecarlo=0.25*(alpha_01/0.03)^2;
number_1=0;
number_greater=0;
loop_1=1;
total_weight_array=zeros(1502,1);
for loop_1=1:1502
number_1=number_1+1;


%bulk carrier
total_bulk_weight=0;
lambda_bulk = 50; 
bulk_U = rand; 
i_1 = 0; 
F_1 = exp(-lambda_bulk); 
while (bulk_U >= F_1); 
F_1 = F_1 + exp(-lambda_bulk)* lambda_bulk^i_1/gamma(i_1+1);
i_1 = i_1 + 1;
end;
bulk_ship_amount=i_1;
for j=1:bulk_ship_amount
intermediate_1=sum( -10 * log(rand(60,1)));
total_bulk_weight=total_bulk_weight+intermediate_1;
end


%container ship
total_container_weight=0;
lambda_container = 25; 
container_U = rand; 
i_2 = 0; 
F_2 = exp(-lambda_container); 
while (container_U >= F_2); 
F_2 = F_2 + exp(-lambda_container)* lambda_container^i_2/gamma(i_2+1);
i_2 = i_2 + 1;
end;
container_ship_amount=i_2;
for j=1:container_ship_amount
intermediate_2=sum( -20 * log(rand(100,1)));
total_container_weight=total_container_weight+intermediate_2;
end


%oil tanker
total_oil_weight=0;
lambda_oil = 25; 
U_oil = rand; 
i_3 = 0; 
F_3 = exp(-lambda_oil); 
while (U_oil >= F_3); 
F_3 = F_3 + exp(-lambda_oil)* lambda_oil^i_3/gamma(i_3+1);
i_3 = i_3 + 1;
end;
oil_ship_amount=i_3;
for j=1:oil_ship_amount
intermediate_3=sum( -50 * log(rand(120,1)) );
total_oil_weight=total_oil_weight+intermediate_3;
end

%comparing results
total_weight=total_oil_weight+total_bulk_weight+total_container_weight;
total_weight_array(number_1)=total_weight;
if total_weight>300000
number_greater=number_greater+1;
end
end

%calculating the required results
expectation = mean(total_weight_array);
deviation = std(total_weight_array);
ratio=number_greater/number_1;
fprintf('Probability: %.4f\n', ratio);
fprintf('Expected value: %.4f\n', expectation);
fprintf('Standard Deviation: %.4f\n', deviation);





\end{lstlisting}
Screenshot:\\
\begin{figure}[H]
  \includegraphics[width=\linewidth]{hw4_ss.png}
  \centering
  \label{fig 2:Pdf with changing sigma values}
\end{figure}
\end{document}